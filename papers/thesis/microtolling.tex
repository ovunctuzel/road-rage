The underlying aim of microtolling is to optimize total travel time among all
drivers. The costs assigned to roads in explanations of Braess' paradox account
for the time to cross the road. The paradox occurs because all rational agents
try to take a new shortcut road with a certain configuration, but if only a few
drivers deviate from the original routes and take the shortcut, shouldn't total
travel time decrease? (I wrote a quick brute-force solver that evaluates total
time for every combination of drivers on each route, and so far, the answer is
no, but there must be a mistake somewhere.) Let the "capacity" of the shortcut
be this critical maximum of drivers that can take the new route and not decrease
the total travel time.

Assuming the critical number of drivers who take the shortcut can be calculated,
the question becomes how to choose the lucky drivers who get the faster route.
A lottery isn't "fair." Microtolling is an alternative. Rational agents minimize
the cost of their route, so to limit the number who take the faster route, costs
should be based on both time and a toll. A new assumption is required: drivers
have different time/dollar ratios; only some drivers are willing to pay a higher
toll for a faster route.

Some questions arise:
1) How does the tolling work?
  a) The shortcut is free/cheap until the capacity is reached, then a high toll
     is imposed. (Unfair because it's first-come, first-serve.)
  b) Everybody pays a toll proportional to the number of drivers on the
     shortcut. If it's empty at first for some driver but more people join while
     that driver is still using it, that driver pays the new high cost. This
     encourages usage at non-peak times.
  c) Once the shortcut is at capacity, do we allow more drivers to join? They
     would get an individual lower travel time, but decrease the total travel
     time of the system. (Is this even possible? Intuitively, yes, because a
     light could remain green for one car to cross quickly, even though the
     other phase has more demand waiting to cross.)
  d) If c) is true, then does an auction work better than a fixed toll?

2) How to generalize to large networks with more than just time costs for roads?
   Intersection costs depend on the flow in connected roads; does this raise any
   fundamental new complication, or can crossing an intersection be modeled as
   just another edge with a cost?

3) Time costs that are just constant or just linear are both unrealistic. The
   time to cross a road is constant (distance / speed limit) until there's
   congestion due to an intersection. This can maybe be modeled by a simple
   piecewise function (constant up to the road's capacity, then some function of
   the number of cars.)


Related work

http://theory.stanford.edu/~tim/papers/nd_hard.pdf
detecting braess paradox and finding the optimal roads to remove is NP-hard

http://economics.mit.edu/files/302
may have done lots of this general analysis already
